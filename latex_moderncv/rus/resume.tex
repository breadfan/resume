\documentclass[12pt,a4paper,sans]{moderncv}
\moderncvtheme[purple]{banking}

\usepackage[scale=0.75]{geometry}

\nopagenumbers{}   

%\usepackage[utf8]{inputenc}
%\usepackage[T2A]{fontenc}             % кодировка исходного текста
\usepackage[russian]{babel}
%\usepackage{lmodern}
\usepackage[unicode]{hyperref}
\definecolor{linkcolour}{rgb}{0,0.2,0.6}
\hypersetup{colorlinks,breaklinks,urlcolor=linkcolour, linkcolor=linkcolour}
%\usepackage{lmodern}
\firstname{Данил}
\familyname{Кизеев}
\address{Санкт-Петербург, Россия}
\mobile{+79040283459}
\email{kizeevdanil@yandex.ru}
%\extrainfo{Software should be beautiful. Both inside and outside.}

\begin{document}
\makecvtitle

\section{Технологии}
\cvline
{}{python, оптимизация, данные, машинное обучение, computer vision, NLP, статистика, numpy, pandas, PostgreSQL, ClickHouse, scipy, transformers, pytorch, C++, Rust, Docker.}
\section{Опыт}

\cventry{2023 Окт -- 2025 Фев}{Инженер ИИ}{ЦИТМ Экспонента + Indeepa}{Москва, Россия}{}{Создал: API по аггрегации и обработке данных, поиск оптимальной цены для продавца, поиск конкурентных товаров, выделение трендов и сезонностей, фреймворк разметки.
	\begin{itemize}
		\item Сбор данных
		\begin{itemize}
			\item Несколько источников: собственный + wb + mpstats;
			\item Исправление ошибок в процессе сбора; дополнение данных для гладкого временного ряда; проводились необходимые аггрегации в зависимости от места торговли;
			\item Итог — создание разных комбинаций источников под разные нужды.
		\end{itemize}
		\item Поиск конкурентов
		\begin{itemize}
			\item Товар переводится в эмбеддинг по изображению, названию и описанию;
			\item Основа на самообученном CLIP на CC12M+COCO переведённые с помощью Helsinki-NLP, внутри которого ResNet и DistilBERT;
			\item В качестве развития сделана сиамская сеть с аггрегацией трешхолдов по категории.
			\item Итог — внедрение в интерфейс SKU потенциальных конкурентов.
		\end{itemize}
		\item Поиск оптимальной цены (ДЦО)
		\begin{itemize}
			\item Решение с помощью бандитов (MAB + CMAB). Модифицированные UCB и Thompson  sampling;
			\item Дисконтирование заказов с учётом тренда + сезонности. НС — BiLSTM + Transformers;
			\item Обучающая выборка — данные продаж + своя модель рынка;
			\item Модель рынка на распределении Пуассона, линейным смещением для тренда и двух гармониках Фурье для сезонности.
		\end{itemize}
		\item Docker + Flask + Gunicorn + Я.Облако для тестов и прода. Выкатка в интерфейс заказчика.
	\end{itemize}
}

\cventry{2022 Сен -- 2023 Мар}{Геймдев-инженер}{Точка сборки}{Пермь, Россия}{}{Создание матчмейкера с помощью методов оптимизации — генетический алгоритм и дифференциальная эволюция, кастомные функции кроссинговера и селекции для эвристических метрик (люди с одинаковыми интересами, одна девушка в команде); создание пайплайнов, визуала и API для сбора данных; визуализации на egui и macroquad; язык Rust; создание игрового движка.}

\cventry{2021 Апр -- 2021 Июн}{Стажер Data Science}{Hyprr}{Санкт-Петербург}{}{
	Автоматическая категоризация постов соц. сети для рекомендательной системы. По тексту (UMAP для сжатия) и изображению из поста получается эмбеддинг (word2vec+tf-idf/BERT + BERTopic), кластеризацией (HDBSCAN, kmeans) с последующим семплингом из категории нового поста определяется похожий и рекомендуется. На второй итерации произведено улучшение (CLIP на ResNet + DistilBERT) для эмбеддинговой части. }



\section{Образование}
\subsection{Университет}
\cventry
{2023 -- 2025}
{Аналитика данных, Магистратура}
{Университет ИТМО, Россия}
{}{}{}
\cventry
{2018 -- 2021}
{Прикладная математика и информатика, Бакалавриат}
{Санкт-Петербургский Государственный Университет, СПбГУ, Россия}
{}{}{}
\cventry
{2017 -- 2018}
{Компьютерная безопасность, Бакалавриат}
{Тверской Государственный Универсистет, ТвГУ, Россия}
{}{}{}




%\subsection{Онлайн-курсы}
%\cventry
%{Апр 2018 - Июн 2018}
%{Алгоритмы: Теория и практика. Методы}
%{}
%{}{\newline\url{https://stepik.org/course/217}}{}
%\cventry
%{Sep 2018 - Nov 2018}
%{Basics of programming and vectorization with R}
%{}
%{}{\newline\url{https://stepik.org/course/497}}{}
%\cventry
%{Sep 2019 - Jan 2020}
%{Elements of financial mathematics}
%{}
%{}{\newline\url{https://intuit.ru/verifydiplomas/101301190}}{}
%\cventry
%{Июн 2020 - Июн 2021}
%{Machine Learning Course}
%{}
%{}{\newline\url{https://mlcourse.ai}}{}
%\cventry
%{Feb 2021 -- Feb 2021}
%{Introduction in geometric programming}
%{}
%{}{\newline\url{https://intuit.ru/verifydiplomas/101428913}}{}

%\cventry
%{2022}
%{Математическая статистика @ CSC}
%{}
%{}{}{}
%\cventry
%{Сен 2022}
%{NLP course @ YSDA}
%{}
%{}{}{}
%\cventry
%{Sep 2022}
%{Deep CV \& Graphics course @ YSDA}
%{}
%{}{}{}





\section{Проекты}
\cvline
{Русско-карельский переводчик}
{\href{https://github.com/breadfan/kielven_project}{Github}\newline{}
	Создал первый онлайн-переводчик для ливвиковского наречия на основе NLLB, mBART и Qwen3. Дополнительно сделал телеграм-бота для разметки данных на параллельный корпус. Сбор данных с помощью собственных парсеров сайта, словаря и википедии. На собранном корпусе в 50к получилось побить существующее решение.
	}
%\cvline
%{Ускоренный МДМ-метод}
%{\href{https://github.com/breadfan/Accelerated_MDM_method}{Github}\newline{}
%	\textbf{Researching acceleration of an MDM-method}\newline{}
%	As a course work for third year two methods were implemented: MDM and accelerated MDM methods with visualization for 2- and 3-dim. Cases for running time comparison.}

  \cvline{Mindmaps for everything}
  {\href{https://github.com/breadfan/mindmaps-for-everything}{Github}\newline{}
  	Создал майндмапы для лучшего понимания и запоминания предметов (вдохновлялся «Цеттелькастеном»). В репозитории есть мапы для статистики, NLP, оптимизации и глубокого обучения, доступны на русском и английском языках.
  }
  \cvline{Рекомендер музыки на основании слов}
  {\href{https://github.com/breadfan/lyrics-based-songs-recommender}{Github}
  	\newline{}
  	Создание эмбеддингов на основании doc2vec и DistilBERT для последющих рекомендаций. Визуализация с помощью UMAP и Plotly. Создал телеграм-бота в качестве оболочки.}
 

  
  	
  


%\section{Навыки}
%\cvline
%  {Языки}{Python, Rust, plpgsql, C/C$++$}
%\cvline
%  {VCS}{Git}
%\cvline
%  {OS}{Windows, Linux (Ubuntu)}




 \subsection{Контакты}
    
    \cvline{GitHub}{\url{https://github.com/breadfan}}
    \cvline{StackOverflow}{\url{https://stackoverflow.com/users/9850300/taciturno}}
    \cvline{LinkedIn}{\url{https://www.linkedin.com/in/rocauc}}
	\cvline{LeetCode}{\url{https://leetcode.com/breadfan/} }
	\cvline{Telegram}{\url{https://t.me/rocauc} }
\end{document}
