\documentclass[11pt,a4paper]{moderncv}

\moderncvtheme[green]{casual}
\usepackage[T1]{fontenc}
\usepackage{lmodern}
\usepackage[utf8]{inputenc}
\usepackage[english, russian]{babel}
\usepackage[scale=0.8]{geometry}

\usepackage[unicode]{hyperref}
\definecolor{linkcolour}{rgb}{0,0.2,0.6}
\hypersetup{colorlinks,breaklinks,urlcolor=linkcolour, linkcolor=linkcolour}

\firstname{Данил}
\familyname{Кизеев}
\address{}{Санкт-Петербург, Россия}
\mobile{+79040283459}
\email{kizeevdanil@yandex.ru}
%\extrainfo{Software should be beautiful. Both inside and outside.}

\begin{document}
	\maketitle
	
	\section{Ключевые слова}
	\cvline
	{}{python, optimization, machine learning, data science, statistics, numpy, pandas, SQL, PostgreSQL, transformers, matplotlib, pytorch, C++, Rust, NLP.}
	
	\section{Опыт}
	\subsection{Полная занятость}
	\cventry{2022 Сен -- 2023 Мар}{Разработчик игр @ Точка сборки}{}{Пермь, Россия}{}{
		Применил генетический алгоритм для построения эффективного матчмейкера для мобильной игры.
		Создал визуализацию работы как приложение с покадровой отрисовкой, используя движок macroquad и egui. Создал пользовательскую и отладочную камеру для 3D компьтерной игры. Сделал методы рэйкастинга для взаимодействия пользователя с объектами. Язык Rust}
	\subsection{Частичная занятость}
	\cventry{2021 Апр -- 2021 Июн}{Data Science стажер}{Под руководством лид DS @ Hyprr}{Санкт-Петербург}{}{
		Решил задачу автоматической категоризации постов пользователей в социальной сети. Сбор и ручная разметка данных. Использование методов NLP для решения задачи из CV}
	\section{Образование}
	\subsection{Университет}
	\cventry
	{2021 -- 2023}
	{Аналитика данных, магистратура}
	{ИТМО}
	{}{}{}
	\cventry
	{2018 -- 2021}
	{Прикладная математика и информатика, бакалавриат}
	{СПбГУ}
	{}{}{}
	\cventry
	{2017 -- 2018}
	{Компьютерная безопасность, специалитет}
	{ТвГУ}
	{}{}{}
	\subsection{Онлайн-курсы, CSC, YSDA}
	\cventry
	{Сен 2022}
	{NLP курс @ YSDA}
	{}
	{}{}{}
	\cventry
	{Сен 2022}
	{Deep CV \& Graphics course @ YSDA}
	{}
	{}{}{}
	\cventry
	{Сен 2022}
	{Системное программирование на Rust @ YSDA}
	{}
	{}{}{}
	\cventry
	{Май 2022}
	{Математическая статистика @ CSC}
	{}
	{}{}{}
	\cventry
	{Feb 2021}
	{Введение в геометрическое программирование}
	{\newline Решение оптимизационных задач с полиномиальной целевой функцией}
	{}{\newline\url{https://intuit.ru/verifydiplomas/101428913}}{}
	\cventry
	{Июн 2021}
	{Machine Learning Course}
	{}
	{}{\newline\url{https://mlcourse.ai}}{}
	\cventry
	{Янв 2020}
	{Элементы финансовой математики}
	{}
	{}{\newline\url{https://intuit.ru/verifydiplomas/101301190}}{}
	\cventry
	{Ноя 2018}
	{Основы программирования и векторизации на R}
	{}
	{}{\newline\url{https://stepik.org/course/497}}{}
	\cventry
	{Июн 2018}
	{Алгоритмы: теория и практика. Методы}
	{}
	{}{\newline\url{https://stepik.org/course/217}}{}
	
	\pagebreak
	\section{Технические навыки}
	\cvline
	{ЯП}{Python, Rust, plpgsql, C/C$++$}
	\cvline
	{VCS}{Git}
	\cvline
	{OS}{Windows, Linux (Ubuntu)}
	\section{Проекты}
	\subsection{Учебные пет проекты}
	\cvline{Майндкарты для всего}
	{\url{https://github.com/breadfan/mindmaps-for-everything}\newline{}
		\textbf{Древовидные диаграммы для объектов}\newline{}
		Создал майндкарты для лучшего понимания и хранения данных об предметах, связанных с математикой и информатикой (по идеям <<Цеттелькастен>>). 
		Присутствуют диаграммы по статистике, NLP, оптимизации и глубокому изучению на PyTorch. Карты доступны на \textbf{русском} и \textbf{английском} языках.  
	}
	\cvline{Рекомендация музыки на основании текстов песен}
	{\url{https://github.com/breadfan/lyrics-based-songs-recommender}\newline{}
		\textbf{Использование doc2vec и BERT эбмбеддингов для рекомендаций основанных на словах.}\newline{}
		Создал модели с d2v, DistilBERT, снизил размерность эмбеддингов с помощью UMAP Для двумерной и трёхмерной визуализаций (Plotly). Создал бота для модели, используя библиотеку <<telegram>>.
	}
	\cvline
	{Автомат. категоризация постов}
	{\url{https://github.com/breadfan/Bachelor-Thesis}\newline{}
		\textbf{Применение BERT для автоматической категоризации постов в социальной сети.} \newline{}
		Использование моделей BERT, BERTopic и word2vec с TF-IDF для категоризации помеченных изображений/видео.\newline{}
		Улучшил качество моделей от BERT к word2vec.\newline{}
		Имея некоторые категории, нужно поставить каждому посту в соответствие категорию для того, чтобы сделать примитивную рекомендательную систему.
	}
	\cvline{База данных банка}
	{\url{https://github.com/breadfan/bank-database}\newline{}
		\textbf{Используя основные фичи plpgsql создать базу данных для внутренних операций.}\newline{}
		Создал функции и процедуры для подсчета кредитного процента, римессы, отрицательных счетов.
		Создал триггеры для добавления новых операций и закрытий счёта. База нормализована по 3 нормальной форме.	  
	}
	\cvline
	{Улучшенный МДМ-метод}
	{\url{https://github.com/breadfan/Accelerated_MDM_method}\newline{}
		\textbf{Имплементация и визуализация МДМ-метода}\newline{}
		В качестве курсовой работы за третий курс были реализованы два метода: МДМ и улучшенный МДМ с визуализации в двумерном и трёхмерном пространствах. Произведено сравнение времени работы алгоритмов.
	}
	
	
	\subsection{В интернете}
	
	\cvline{GitHub}{\url{https://github.com/breadfan}}
	\cvline{StackOverflow}{\url{https://stackoverflow.com/users/9850300/taciturno}}
	\cvline{LinkedIn}{\url{https://www.linkedin.com/in/danil-kizeev-57a48bba}}
	\cvline{LeetCode}{\url{https://leetcode.com/breadfan/} }
\end{document}