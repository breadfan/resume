\documentclass[11pt,a4paper]{moderncv}

\moderncvtheme[green]{casual}
\usepackage[utf8]{inputenc}
\usepackage[scale=0.8]{geometry}

\usepackage[unicode]{hyperref}
\definecolor{linkcolour}{rgb}{0,0.2,0.6}
\hypersetup{colorlinks,breaklinks,urlcolor=linkcolour, linkcolor=linkcolour}
\usepackage[T1]{fontenc}
\usepackage{lmodern}
\firstname{Danil}
\familyname{Kizeev}
\address{Saint Petersburg, Russia}
\mobile{+79040283459}
\email{kizeevdanil@yandex.ru}
%\extrainfo{Software should be beautiful. Both inside and outside.}

\begin{document}
\maketitle

\section{Keywords}
\cvline
{}{python, optimization, machine learning, data science, computer vision, NLP, statistics, numpy, pandas, PostgreSQL, ClickHouse, scipy, transformers, pytorch, C++, Rust, Docker.}
\section{Experience}

\cventry{2021 Apr -- 2021 Jun}{Data Science intern @ Hyprr}{}{Saint-Petersburg}{}{Solved case of automatic categorization of users posts in social network. Scraped, labeled the data. On the first iteration used Google API for getting images descriptions. On the second iteration used CLIP over ResNet and DistilBERT for the posts, KMeans for clusterization and mapping objects.}
\cventry{2022 Sep -- 2023 Mar}{Game developer @ Hive}{}{Perm, Russia}{}{Applying genetic and differential evolution algorithms to build an efficient matchmaker for multiplayer in online mobile games; making visualizations via egui and macroquad; building data collection pipelines; rust language. Making player and debug cameras; building raycasting.}
\cventry{2023 Oct -- 2025 Feb}{AI Engineer @ ECMC Exponenta}{}{Moscow, Russia}{}{
	Working with the problem of searching competitors to the main product. Made our own labeling framework using telegram bot API. Trained CLIP for russian language using CC12M+COCO+translations datasets+own translations using Helsinki-NLP models. Used CLIP over ResNet and DistilBERT. Using the resulting labels trained siamese model and aligned it to classification problem. Made models with FashionCLIP and XGBoost as a head for classification, compared them. 
	Implementing optimal prices models using contextual multi-armed bandits (CMAB). Creating models of trend, seasonality, demand using time-series models (Prophet, XGB, xARIMAx, linear) and neural networks (BiLSTM, Transformers). Applying to production with customer. Lifting using Docker+Flask for MVP's}

\section{Education}
\subsection{University}
\cventry
{2017 -- 2018}
{Cybersecurity department, Bachelor}
{Tver State University, TVGU, Russia}
{}{}{}
\cventry
{2018 -- 2021}
{Applied mathematics and computer science, Bachelor}
{Saint-Petersburg State University, SPBGU, Russia}
{}{}{}
\cventry
{2023 -- 2025}
{Data Analytics, Master}
{ITMO University, Russia}
{}{}{}


\subsection{MOOCs, CSC, YSDA}
\cventry
{Apr 2018 - Jun 2018}
{Algorithms: theory and practice. Methods}
{}
{}{\newline\url{https://stepik.org/course/217}}{}
%\cventry
%{Sep 2018 - Nov 2018}
%{Basics of programming and vectorization with R}
%{}
%{}{\newline\url{https://stepik.org/course/497}}{}
%\cventry
%{Sep 2019 - Jan 2020}
%{Elements of financial mathematics}
%{}
%{}{\newline\url{https://intuit.ru/verifydiplomas/101301190}}{}
\cventry
{Jun 2020 - Jun 2021}
{Machine Learning Course}
{}
{}{\newline\url{https://mlcourse.ai}}{}
%\cventry
%{Feb 2021 -- Feb 2021}
%{Introduction in geometric programming}
%{}
%{}{\newline\url{https://intuit.ru/verifydiplomas/101428913}}{}

\cventry
{2022}
{Mathematical statistics @ CSC,}
{}
{}{}{}
\cventry
{Sep 2022}
{NLP course @ YSDA,}
{}
{}{}{}
\cventry
{Sep 2022}
{Deep CV \& Graphics course @ YSDA}
{}
{}{}{}









\pagebreak
\section{Projects}
  \subsection{Study projects/pet projects}
  \cvline{Mindmaps for everything}
  {\url{https://github.com/breadfan/mindmaps-for-everything}\newline{}
  	\textbf{Smart diagrams for math-related objects}\newline{}
  	Created mindmaps for better understanding ("Zettelkasten" inspiration) and keeping in mind tech-related objects. There are statistics, NLP, optimization and deep learning using pytorch now. Maps are available in \textbf{rus} and \textbf{eng} languages.	  
  }
  \cvline{Lyric music recommender}
  {\url{https://github.com/breadfan/lyrics-based-songs-recommender}\newline{}
  	\textbf{Using doc2vec and BERT embeddings for recommendation based on songs lyrics.}\newline{}
  	Created models with doc2vec and DistilBERT, created UMAP reduced embeddings for 2-d and 3-d visualisations (Plotly). Generated bot using <<telegram>> library.	  
  }
	\cvline
	{Automatic posts categorization}
	{\url{https://github.com/breadfan/Bachelor-Thesis}\newline{}
		\textbf{Applying BERT for automatic posts categorization in social network.} \newline{}
		Using BERT, BERTopic and word2vec + TF-IDF for labeled images/video categorization. \newline{}
		Upgraded quality of models from BERT to word2vec.\newline{}
		Having categories need to make mapping from labels amount of posts to that categories for making simple recommendations.
  }
  \cvline
  {Accelerated MDM-method}
  {\url{https://github.com/breadfan/Accelerated_MDM_method}\newline{}
  	\textbf{Researching acceleration of an MDM-method}\newline{}
  	As a course work for third year two methods were implemented: MDM and accelerated MDM methods with visualization for 2- and 3-dim. Cases for running time comparison.}
	\cvline
	{Russian-Karelian translator}
	{\url{https://github.com/breadfan/kielven_project}\newline{}
		\textbf{Creating translator for livvi based on NLLB, mBART and Qwen3. Creating framework for parallel corpora labelling}\newline{}
		As a masters degree created first online karelian translator. Elevated Meta's results, as well as Alibaba (for Qwen). Parsed corpora from Russian-Karelian dictionary. Parsed olo.wikipedia.org. Parsed vepkar.ru. Made frameword for corpora parallel labelling.}
  	
  


\section{Technical Skills}
\cvline
  {Languages}{Python, Rust, plpgsql, C/C$++$}
\cvline
  {VCS}{Git}
\cvline
  {OS}{Windows, Linux (Ubuntu)}




 \subsection{On the Internet}
    
    \cvline{GitHub}{\url{https://github.com/breadfan}}
    \cvline{StackOverflow}{\url{https://stackoverflow.com/users/9850300/taciturno}}
    \cvline{LinkedIn}{\url{https://www.linkedin.com/in/rocauc}}
	\cvline{LeetCode}{\url{https://leetcode.com/breadfan/} }
	\cvline{Telegram}{\url{https://t.me/rocauc} }
\end{document}
