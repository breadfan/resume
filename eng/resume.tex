\documentclass[11pt,a4paper]{moderncv}

\moderncvtheme[green]{casual}
\usepackage[utf8]{inputenc}
\usepackage[scale=0.8]{geometry}

\usepackage[unicode]{hyperref}
\definecolor{linkcolour}{rgb}{0,0.2,0.6}
\hypersetup{colorlinks,breaklinks,urlcolor=linkcolour, linkcolor=linkcolour}

\firstname{Danil}
\familyname{Kizeev}
\address{}{Saint Petersburg, Russia}
\mobile{+79040283459}
\email{kizeevdanil@yandex.ru}
%\extrainfo{Software should be beautiful. Both inside and outside.}

\begin{document}
\maketitle

\section{Keywords}
\cvline
{}{python, optimization, machine learning, data science, statistics, numpy, pandas, PostgreSQL, ClickHouse, scipy, transformers, pytorch, C++, Rust, NLP, Docker.}
\section{Experience}

\subsection{Actual}
\cventry{2022 Sep -- 2023 Mar}{Game developer @ Hive}{}{Perm, Russia}{}{Applying genetic and differential evolution algorithms to build an efficient matchmaker for multiplayer in online mobile games; making visualizations via egui and macroquad; building data collection pipelines; rust language. Making player and debug cameras; building raycasting.}
\subsection{Vocational}
\cventry{2021 Apr -- 2021 Jun}{Data Science intern}{Lead by lead DS @ Hyprr}{Saint-Petersburg}{}{Solved case of automatic categorization of users posts in social network. Scraped, labeled the data. Used NLP for CV domain.}

\section{Education}
\subsection{University}
\cventry
{Sep 2021 -- Jun 2024}
{Data Analytics, Master}
{Saint-Petersburg National Research University of Information Technologies, Mechanics and Optics, Russia}
{}{}{}
\cventry
{Sep 2018 -- Jun 2021}
{Applied mathematics and computer science, Bachelor}
{Saint-Petersburg State University, Russia}
{}{}{}
\cventry
{2017 -- 2018}
{Cybersecurity department, Bachelor}
{Tver State University, Russia}
{}{}{}
\subsection{MOOCs, CSC, YSDA}
\cventry
{Sep 2022}
{NLP course @ YSDA}
{}
{}{}{}
\cventry
{Sep 2022}
{Deep CV \& Graphics course @ YSDA}
{}
{}{}{}
\cventry
{Feb 2022 -- May 2022}
{Mathematical statistics @ CSC}
{}
{}{}{}
\cventry
{Feb 2021 -- Feb 2021}
{Introduction in geometric programming}
{}
{}{\newline\url{https://intuit.ru/verifydiplomas/101428913}}{}
\cventry
{Jun 2020 - Jun 2021}
{Machine Learning Course}
{}
{}{\newline\url{https://mlcourse.ai}}{}
\cventry
{Sep 2019 - Jan 2020}
{Elements of financial mathematics}
{}
{}{\newline\url{https://intuit.ru/verifydiplomas/101301190}}{}
\cventry
{Sep 2018 - Nov 2018}
{Basics of programming and vectorization with R}
{}
{}{\newline\url{https://stepik.org/course/497}}{}

\cventry
{Apr 2018 - Jun 2018}
{Algorithms: theory and practice. Methods.}
{\newline Learned basics of algorithms and practiced it}
{}{\newline\url{https://stepik.org/course/217}}{}

\pagebreak
\section{Projects}
  \subsection{Study projects/pet projects}
  \cvline{Mindmaps for everything}
  {\url{https://github.com/breadfan/mindmaps-for-everything}\newline{}
  	\textbf{Smart diagrams for math-related objects}\newline{}
  	Created mindmaps for better understanding ("Zettelkasten" inspiration) and keeping in mind tech-related objects. There are statistics, NLP, optimization and deep learning using pytorch now. Maps are available in \textbf{rus} and \textbf{eng} languages.	  
  }
  \cvline{Lyric music recommender}
  {\url{https://github.com/breadfan/lyrics-based-songs-recommender}\newline{}
  	\textbf{Using doc2vec and BERT embeddings for recommendation based on songs lyrics.}\newline{}
  	Created models with doc2vec and DistilBERT, created UMAP reduced embeddings for 2-d and 3-d visualisations (Plotly). Generated bot using <<telegram>> library.	  
  }
	\cvline
	{Automatic posts categorization}
	{\url{https://github.com/breadfan/Bachelor-Thesis}\newline{}
		\textbf{Applying BERT for automatic posts categorization in social network.} \newline{}
		Using BERT, BERTopic and word2vec + TF-IDF for labeled images/video categorization. \newline{}
		Upgraded quality of models from BERT to word2vec.\newline{}
		Having categories need to make mapping from labels amount of posts to that categories for making simple recommendations.
  }
  \cvline
  {Accelerated MDM-method}
  {\url{https://github.com/breadfan/Accelerated_MDM_method}\newline{}
  	\textbf{Researching acceleration of an MDM-method}\newline{}
  	As a course work for third year two methods were implemented: MDM and accelerated MDM methods with visualization for 2- and 3-dim. Cases for running time comparison.}

  	
  


\section{Technical Skills}
\cvline
  {Languages}{Python, Rust, plpgsql, C/C$++$}
\cvline
  {VCS}{Git}
\cvline
  {OS}{Windows, Linux (Ubuntu)}




  \subsection{On the Internet}
    
    \cvline{GitHub}{\url{https://github.com/breadfan}}
    \cvline{StackOverflow}{\url{https://stackoverflow.com/users/9850300/taciturno}}
    \cvline{LinkedIn}{\url{https://www.linkedin.com/in/rocauc}}
	\cvline{LeetCode}{\url{https://leetcode.com/breadfan/} }
\end{document}
